\documentclass{article}

\usepackage{amsmath}

\begin{document}

The 2D navier-stokes equations for a newtonian fluid are:

\begin{align}
\partial_t \rho + \partial_i \left( \rho u_i \right) & = 0 \\
\partial_t \left( \rho u_i \right) + \partial_j \left( \rho u_i u_j \right)
	& = - \partial_i p + \partial_j \tau_{ij} + \rho g_i\\
\tau_{ij} & = \mu \left( \partial_i u_j + \partial_j u_i \right)
\end{align}

Suppose our experiment is taking place in a thin region between two
	flat frictionless plates, and that the plates are placed horizontally.
If our velocities are not too high, then the material properties are not
	expected to change drastically, and we can thus assume constant
	density and constant viscosity.
Since our experiment is between two plates, all velocities are two-dimensional
	velocities.
Since our plates are placed horizontally, we can neglect all body forces.
Similarly, since our plates are placed horizontally, and the region
	of interest is very thin, and we assume that a constant pressure
	from the atmosphere is applied on both sides of the plates,
	then we can assume that the pressure within the region of 
	interest is constant.

If we assume constant density, we can simplify the mass-conservation
	equations drastically, and we can also simplify the momentum 
	equation by dividing by density. 
When density is constant, the mass conservation equation simply becomes
	$\partial_i u_i = 0$.

This will also have an impact on the momentum equation: note that 
	$\partial_j \left( u_i u_j \right) =
	\left( \partial_j u_i \right) u_j 
	+ u_i \left( \partial_j u_j \right)$.
Since we found that constant density implied that $\partial_j u_j = 0$,
	we can drop the second term and just write $u_j \partial_j u_i$.

If we assume constant viscosity, we can move the viscosity out of the 
	derivative of the deviatoric stress $\tau$.
Now, $\partial_j \tau_{ij} = \mu \left( \partial_i 
	\left( \partial_j u_j \right) + \partial_j \partial_j u_i \right)$.
Since we found that constant density implied $\partial_j u_j = 0$,
	we can thus conclude that $\partial_j \tau_{ij} 
	= \mu \partial_j \partial_j u_i$.

If we neglect body forces, we can drop the $\rho g_i$ term.
Similarly if we neglect pressure forces, we can drop the $\partial_i p$ term.

The navier-stokes equations now become ($\nu = \frac{\mu}{\rho}$):

\begin{align}
\partial_i u_i & = 0 \\
\partial_t u_i + u_j \partial_j u_i & = \nu \nabla^2 u_i\\
\end{align}



\end{document}
