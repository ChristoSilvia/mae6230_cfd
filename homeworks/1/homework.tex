\documentclass{article}

\usepackage{amsmath}

\DeclareMathOperator\erf{erf}

\begin{document}

\section{Partners}



\section{Free Shear Flows}

In this section, we investigate two types of 2D free shear flows:
	the splitter-plate and the wake.
In the splitter-plate, two flows of different velocity are separated by
	a splitter, until they meet and mix.
In the wake, interactions with an obstacle cause a region of reduced
	velocity behind an object, and this region of reduced
	velocity spreads and diffuses as the fluid continues onwards.

\subsection{Deriving the Governing Equations}

To begin our analysis of 2D shear flows, we need to determine exactly 
	which partial differential equations and boundary conditions
	we need to use.

\subsubsection{Simplifying Navier-Stokes for 2D free shear flow}

The 2D navier-stokes equations for a newtonian fluid are:

\begin{align}
\partial_t \rho + \partial_i \left( \rho u_i \right) & = 0 \\
\partial_t \left( \rho u_i \right) + \partial_j \left( \rho u_i u_j \right)
	& = - \partial_i p + \partial_j \tau_{ij} + \rho g_i\\
\tau_{ij} & = \mu \left( \partial_i u_j + \partial_j u_i \right)
\end{align}

Suppose our experiment is taking place in a thin region between two
	flat frictionless plates, and that the plates are placed horizontally.
If our velocities are not too high, and the material properties are not
	expected to change drastically, we can thus assume constant
	density and constant viscosity.
Since our experiment is between two plates, all velocities are two-dimensional
	velocities.
Since our plates are placed horizontally, with a very small height,
	the body forces are not going to cause a significant difference
	in pressure, and can thus be neglected.
Similarly, since our plates are placed horizontally, and the region
	of interest is very thin, and we assume that a constant pressure
	from the atmosphere is applied on both sides of the plates,
	then we can assume that the pressure within the region of 
	interest is constant.

If we neglect body forces, we can drop the $\rho g_i$ term.
Similarly if we neglect pressure forces, we can drop the $\partial_i p$ term.

If we assume constant density, we can simplify the mass-conservation
	equations drastically, and we can also simplify the momentum 
	equation by dividing by density. 
When density is constant, the mass conservation equation simply becomes
	$\partial_i u_i = 0$.

This will also have an impact on the momentum equation. 
The convection term is given by the following expression:
	$\partial_j \left( u_i u_j \right) =
	\left( \partial_j u_i \right) u_j 
	+ u_i \left( \partial_j u_j \right)$.
Since we found that constant density implied that $\partial_j u_j = 0$,
	we can drop the second term and just write $u_j \partial_j u_i$.

If we assume constant viscosity, we can move the viscosity out of the 
	derivative of the deviatoric stress $\tau$.
Now, $\partial_j \tau_{ij} = \mu \left( \partial_i 
	\left( \partial_j u_j \right) + \partial_j \partial_j u_i \right)$.
Since we found that constant density implied $\partial_j u_j = 0$,
	we can thus conclude that $\partial_j \tau_{ij} 
	= \mu \partial_j \partial_j u_i$.

The navier-stokes equations now become ($\nu = \frac{\mu}{\rho}$):

\begin{align}
\partial_i u_i & = 0 \\
\partial_t u_i + u_j \partial_j u_i & = \nu \nabla^2 u_i
\end{align}

\subsubsection{Boundary conditions}

The boundary conditions for our flows are different in the temporally-
	evolving limit than they are in the rest frame.

In the rest frame, for the splitter-plate, the boundary conditions are that
	there is a no-slip condition at the splitter-plate.
Before the splitter plate, on either side, there are inlet boundary
	conditions with a fixed velocity.
Below and above the splitter plate, the boundary conditions are that
	infinitely below the splitter plate the velocity
	assumes the value of the velocity in the lower inlet,
	and infinitely above the splitter plate the velocity
	assumes the value of the velocity in the upper inlet.
There is no boundary condition far to the right of the region of interest,
	i.e., it is an "exit" boundary condition.

In the rest frame, for the wake, we assume that the object producing the
	wake is not in the region of interest.
We then have to the have the inlet boundary condition, to the left,
	be the velocity profile downstream of an object.
The upper and lower boundary conditions, at effective infinity, 
	are that the velocity is the stream velocity.
The right boundary conditions is an "exit" boundary condition.

For the moving frames, where the spatial to temporal approximation is made,
	the boundary conditions are similar but different.
For both situations, uniformity in $x$ is assumed, so that we only
	have to worry about variations in $y$.

For the temporally-evolving splitter-plate case, our boundary condition
	is, if the velocity deficit between the two streams is $2 U$,
	the velocity at $y=-\infty$ is $-U$, and the velocity at
	$y = \infty$ is $U$.

For the temporally-evolving wake case, our boundary conditions is that
	the velocity deficit goes to $0$ at both $y = -\infty$ and 
	at $y=\infty$.

\subsection{Mixing Layer Solution}

By deciding to move in a reference frame which moves with the average fluid
	velocity, the mixing layer problem is simplified.
Instead of the two fluids moving with large velocities, with their velocities
	being slightly different, in the moving reference frame the two fluids 
	move in opposite directions.
The faster fluid moves with velocity $U\hat{x}$ in the moving
	frame, and the slower fluid moves with velocity $-U\hat{x}$.
The price we pay for this misdirection is that the relative fluid velocities
	must be small.
An intuitive reason for this being the case is that for the mixing
	to be time-evolving, the two fluids must be carried away from the
	splitter far faster then the timescale on which they mix.

Since we're in a moving reference frame, the spatially-and-temporally
	propagating problem from before now reduces just to a temporally
	evolving system.
Since the fluid is moving much faster than the relative velocities,
	we can assume that any non-uniformities along the direction
	of flow are on much larger scales than in the direction
	between the two fluids.
We can therefore assume that the fluid is identical in the $x$ 
	direction.

We assume that at time $t=0$, when the fluids just pass the splitter-plate,
	the fluid above $y=0$ is all moving with velocity $U \hat{x}$,
	and the fluid below $y=0$ is all moving with velocit $-U \hat{x}$.
Since the velocities are small, as assumed before, \emph{we can neglect the
	convective term}, which is quadratic in velocity.
Since the velocity we care about is the velocity in the $\hat{x}$ direction,
	and we only see it varying along $y$, we can write this as 
	$u(y)$ and consider only the diffusion of $u$ along $y$ over time:

\begin{align}
\frac{\partial u}{\partial t} & = \nu \frac{\partial^2 u}{\partial y^2}
\end{align}

As for initial conditions, we assume that because of whatever physics
	occurs at the splitter plate, the initial conditions are
	not an exact step function, but follow the following 
	profile:

\begin{align}
u(y,t=0) & = U \erf \left( \frac{ \sqrt{\pi} y}{\delta } \right)
\end{align}

We will attack this problem with the Fourier transform.
We will use the convention where $\mathcal{F}(f(x))(\xi) =
	 \int f(x) e^{-2 \pi i \xi y} dx$.
With this convention, $\mathcal{F} \left(\partial_y u(y)\right) 
	= 2\pi i \xi \hat{u}(\xi)$
Thus, $\mathcal{F} \left(\partial_y^2 u(y)\right) 
	= - 4 \pi^2 \xi^2 \hat{u}(\xi)$.

We can now write the PDE, transformed to $\xi$-space, and its solution
	for different values of $t$:

\begin{align}
\partial_t \hat{u}(\xi, t) & = - 4 \pi^2 \nu \xi^2 \hat{u}(\xi, t) \\
\hat{u}(\xi, t) & = \exp \left( - 4 \pi^2 \xi^2 \nu t \right) \hat{u}(\xi, 0)
\end{align}

Since we know $u(x,0)$, we can find $\hat{u}(\xi, 0)$.
Since we know the fourier transform of $f(a x)$ if we know the
	fourier transform of $f(x)$, we can find the fourier
	transform of $f(a x)$. 

\begin{align}
\mathcal{F}(\erf(x))(\xi) & = 
	\frac{-i}{\pi \xi} \exp \left(\pi^2 \xi^2 \right)\\
\mathcal{F}(f(a x))(\xi) & = 
	\frac{1}{a} \mathcal{F}(f(x))\left(\frac{\xi}{a}\right)\\
\mathcal{F}(\erf(a x))(\xi) & =
	\frac{-i}{\pi \xi} \exp \left(\pi^2 
		\left( \frac{\xi}{a} \right)^2 \right)
\end{align}

Since $u(x,0) = U \erf \left( \frac{\sqrt{\pi}}{\delta} y \right) $,
	the fourier transform of $u(x, 0)$ is:

\begin{align}
\hat{u}(\xi, 0) & = u
	\frac{- i }{\pi \xi}\exp \left(-\pi^2 
		\left( \frac{\xi}{\frac{\sqrt{\pi}}{\delta}} \right)^2 \right)\\
\hat{u}(\xi, t) & = 
	\exp \left( - 4 \pi^2 \xi^2 \nu t \right) \hat{u}(\xi, 0) \nonumber \\
& = U
	\frac{- i }{\pi \xi}\exp \left(-\pi^2 \xi^2 
		\left( \frac{\delta^2}{\pi}
		+ 4 \nu t \right) \right) \nonumber\\
\hat{u}(\xi, t) &  = U \frac{- i}{\pi \xi} 
	\exp\left(- \pi^2 \left( \frac{\xi}{\frac{\sqrt{\pi}}{
	\sqrt{\delta^2 + 4 \pi \nu t}}} \right)^2 \right)
\end{align}

We can do an inverse fourier transform, using the formula for the fourier
	transform of $\erf$, to get:

\begin{align}
u(y,t) & = U \erf \left( y \frac{\sqrt{\pi}}
	{\sqrt{\delta^2 + 4 \pi \nu t}} \right) \nonumber \\
u(y,t) & = U \erf \left( \frac{ \sqrt{\pi} y}{\delta} 
	\sqrt{ \frac{\delta^2}{\delta^2 + 4 \pi \nu t}} \right) 
\end{align}

\subsubsection{Vorticity Thickness}

We can define the vorticity thickness $\delta_w$ as:

\begin{align}
\delta_w & = \frac{\Delta U}{\left( \partial u / \partial y \right)_\text{max}}
\end{align}

The derivative of $\erf$ is $\partial_x \erf(x) 
	= \frac{2}{\sqrt{\pi}} e^{-x^2}$.
Thus:

\begin{align}
\partial u / \partial y & = \frac{2 U}{\sqrt{\pi}} 
	\exp \left( - \left( \frac{ \sqrt{\pi} y}{\delta} 
	\sqrt{ \frac{\delta^2}{\delta^2 + 4 \pi \nu t}} \right)^2 \right) 
	\frac{ \sqrt{\pi}}{\delta} 
	\sqrt{ \frac{\delta^2}{\delta^2 + 4 \pi \nu t}} \nonumber \\
\left( \partial u / \partial y \right)_\text{max} & =
	2 U \frac{1}{\delta} \frac{1}{\sqrt{1 + \frac{4 \pi \nu}{\delta^2} t}}\\
\delta_w & = \delta \sqrt{ 1 + \left( \frac{4 \pi \nu}{\delta^2} \right) t } 
\end{align}

We can see that in the limit of zero viscosity, $\delta_w = \delta$.

\subsection{Wake}

To solve the wake problem, we will follow the same general procedure
	as before.

\begin{align}
u(y,0) & = -U\exp\left(-\frac{\pi y^2}{\delta^2} \right)\\
\hat{u}(\xi, 0) & = - U \sqrt{ \frac{\pi}{\frac{\pi}{\delta^2}}}
	\exp \left( - \frac{ (\pi \xi)^2}{ \frac{\pi}{\delta^2}} \right)
	\nonumber \\
\hat{u}(\xi, 0)& = - U \delta \exp \left( - 
	\pi \left( \delta \xi \right)^2 \right) \\	
\hat{u}(\xi, t) & = - U \delta 
	\exp \left( - \pi \left( \delta \xi \right)^2 \right)
	\exp \left( - 4 \pi^2 \xi^2 \nu t \right) \nonumber\\
\hat{u}(\xi, t) & = - U \delta
	\sqrt{ \frac{\frac{1}{ \frac{\delta^2}{\pi} + 4 \nu t}}{\pi}}
	\sqrt{ \frac{\pi}{ \frac{1}{ \frac{\delta^2}{\pi} + 4 \nu t}}}
	\exp \left( - \frac{ ( \pi \xi )^2}
	{ \frac{1}{ \frac{\delta^2}{\pi} + 4 \nu t}} \right) \nonumber\\
u(y, t) & = - U \sqrt{ \frac{\delta^2}{ \delta^2 + 4 \pi \nu t}}
	\exp \left( \frac{-\pi y^2}{ \delta^2 + 4 \pi \nu t} \right)
\end{align}

The wake half-width $b$ is defined as the width between the centerline
	and the point where the velocity profile drops to half its
	maximum deficit value.
If the velocity profile is proportional to $\exp(-a x^2)$, this point occurs
	where $\exp(-a b^2) = 1/2$.  
This occurs at $b = \sqrt{\frac{\log 2}{a}}$.
In the velocity profile above, $a(t) = \frac{\pi}{\delta^2 + 4 \pi \nu t}$.
Thus:

\begin{align}
b(t) & = \sqrt{ \frac{ \log 2}{\pi} \left( \delta^2 + 4 \pi \nu t \right) }
\end{align}

In the case of no viscosity, $b = \delta \sqrt{\frac{\log 2}{\pi}}$

\end{document}
