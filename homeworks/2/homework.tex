\documentclass{article}

\usepackage{amsmath}

\begin{document}

\section{Lagrange Interpolating Polynomial}

We are trying to write a polynomial which, if we are considering
	the points $x_0, \dots, x_n$, is equal to one at $x_i$,
	and equal to zero for all $x_j$, with $j \neq i$.
We exhibit this polynomial below.

\begin{align}
f_i(x) & = \prod_{j \neq i} \frac{x - x_j}{x_i - x_j}
\end{align}

Note that $f_i(x_i) = \prod_{j \neq i} \frac{x_i - x_j}{x_i - x_j}$,
	and since all of the numerators and denominators are equal,
	we can see that $f_i(x_i) = 1$.
For $f_i(x_k) = \prod_{j \neq i} \frac{x_k - x_j}{x_i - x_j}$,
	if $k \neq i$, since $j$ ranges over all indices except $i$,
	one of the $j$'s will be equal to $k$.
That will make the numerator zero, and thus the whole product
	will be zero.
Therefore, $f_i(x_k) = \delta_{ik}$.

If we want our polynomial to have the value $y_i$ at each
	point $x_i$, we can sum several of these polynomials, so that
	the resultant polynomial has the characteristics which we desire:

\begin{align}
f(x) & = \sum_i y_i f_i(x) \\
& = \sum_i y_i 
	\left( \prod_{j \neq i} \frac{x-x_j}{x_i-x_j} \right)
\end{align}

This polynomial has the required values at each point.

\section{Differentiating a Lagrange Interpolating Polynomial}

Given the definition of $f_i(x)$ given above, we can compute
	the derivative of $f_i(x)$ with respect to $x$.

\begin{align}
\frac{d f_i(x)}{dx} & = \sum_{\begin{matrix}k=0\\k\neq i\end{matrix}}^n 
	\frac{1}{x_i - x_k}
	\left( \prod_{
		\begin{matrix} j = 0 \\ j \neq k\\j \neq i \end{matrix}}^{j=n}
		\frac{x - x_j}{x_i - x_j} \right)
\end{align}

Therefore, for the whole approximating function $f(x)$, we have:

\begin{align}
\frac{df}{dx}(x) & = \sum_{\begin{matrix}i=0\end{matrix}}^n
	y_i
	\sum_{\begin{matrix}k=0\\k\neq i\end{matrix}}^n
	\frac{1}{x_i - x_k}
	\left( \prod_{
		\begin{matrix} j = 0 \\ j \neq k\\j \neq i \end{matrix}}^{j=n}
		\frac{x - x_j}{x_i - x_j} \right)
\end{align}

This can be interpeted as a dot product.
If we consider the vector $\vec{D}(x)$ given by:

\begin{align}
D_i(x) & = 
	\sum_{\begin{matrix}k=0\\k\neq i\end{matrix}}^n
	\frac{1}{x_i - x_k}
	\left( \prod_{
		\begin{matrix} j = 0 \\ j \neq k\\j \neq i \end{matrix}}^{j=n}
		\frac{x - x_j}{x_i - x_j} \right)
\end{align}

Then if we consider the vector $\vec{y} = \{ y_0, \dots, y_n \}$,
	then $\frac{df}{dx}(x) = \vec{D}(x) \cdot \vec{y}$.

There's a special case to this.  
Suppose $x = x_l$ is one of the coordinates.
Then, $D_i(x_l)$ is given by:

\begin{align}
D_i(x_l) & = 
	\sum_{\begin{matrix}k=0\\k\neq i\end{matrix}}^n
	\frac{1}{x_i - x_k}
	\left( \prod_{
		\begin{matrix} j = 0 \\ j \neq k\\j \neq i \end{matrix}}^{j=n}
		\frac{x_l - x_j}{x_i - x_j} \right)
\end{align}

However, this can be greatly simplified.
If $x_l = x_i$, then we see that the product terms all drop out,
	since the numerators and denomiators are all equal.
If $x_l \neq x_i$, then the product would only be nonzero for the
	case where $k=l$.
Therefore, only that term in the sum survives.
We show the special cases for $D_i(x_l)$ below:

\begin{align}
D_i(x_i) & = \sum_{\begin{matrix}k=0\\k\neq i\end{matrix}} \frac{1}{x_i - x_k}\\
\begin{matrix}D_i(x_l)\\l \neq i\end{matrix} & = \frac{1}{x_i - x_l} 
	\prod_{
		\begin{matrix} j = 0 \\ j \neq l\\j \neq i \end{matrix}}^{j=n}
		\frac{x_l - x_j}{x_i - x_j} 
\end{align}

\section{Placeholder Title for Problem 3}

\section{Placeholder Title for Problem 4}

\section{Computation of mesh}



\end{document}
