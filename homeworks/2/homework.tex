\documentclass{article}

\usepackage{amsmath}

\begin{document}

\section{Lagrange Interpolating Polynomial}

We are trying to write a polynomial which, if we are considering
	the points $x_0, \dots, x_n$, is equal to one at $x_i$,
	and equal to zero for all $x_j$, with $j \neq i$.
We exhibit this polynomial below.

\begin{align}
p_i(x) & = \prod_{j \neq i} \frac{x - x_j}{x_i - x_j}
\end{align}

Note that $p_i(x_i) = \prod_{j \neq i} \frac{x_i - x_j}{x_i - x_j}$,
	and since all of the numerators and denominators are equal,
	we can see that $p_i(x_i) = 1$.
For $p_i(x_k) = \prod_{j \neq i} \frac{x_k - x_j}{x_i - x_j}$,
	if $k \neq i$, since $j$ ranges over all indices except $i$,
	one of the $j$'s will be equal to $k$.
That will make the numerator zero, and thus the whole product
	will be zero.
Therefore, $p_i(x_k) = \delta_{ik}$.

If we want our polynomial to have the value $f(x_i)$ at each
	point $x_i$, we can sum several of these polynomials, so that
	the resultant polynomial has the characteristics which we desire:

\begin{align}
p(x) & = \sum_i f(x_i) 
	\left( \prod_{j \neq i} \frac{x-x_j}{x_i-x_j} \right)
\end{align}

This polynomial has the required values at each point.

\section{Differentiating a Lagrange Interpolating Polynomial}

\end{document}
